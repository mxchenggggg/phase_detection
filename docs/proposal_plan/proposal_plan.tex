\documentclass[11pt]{article} \usepackage[top=1in, bottom=1in, left=1in, right=1in]{geometry}
\usepackage{amsfonts, amsmath, amssymb, amsthm}
\usepackage{xcolor}
\usepackage{hyperref}
\usepackage[british,calc]{datetime2}
\usepackage{advdate}
\usepackage{tikz}
\usetikzlibrary{calc,arrows}

% Timeline https://tex.stackexchange.com/questions/61237/timeline-and-tikz
\newdimen\XCoord
\newdimen\YCoord
\newcommand*{\ExtractCoordinate}[1]{\path (#1); \pgfgetlastxy{\XCoord}{\YCoord};}
% modify these for altered timeline appearance
% ============================= 
\pgfmathsetmacro{\mintime}{0}
\pgfmathsetmacro{\maxtime}{14}
\newcommand{\timeunit}{Week}
\pgfmathtruncatemacro{\timeintervals}{14}
\pgfmathsetmacro{\scaleitemseparation}{3}
\pgfmathsetmacro{\timenodewidth}{8cm}
\newcounter{itemnumber}
\setcounter{itemnumber}{0}
\newcommand{\lastnode}{n-0}
% ============================= 
% entry in timeline
\newcommand{\timeentry}[2]{% time, description
\stepcounter{itemnumber}
\node[below right,text width=\timenodewidth] (n-\theitemnumber) at (\lastnode.south west) {#2};
\edef\lastnode{n-\theitemnumber}
\expandafter\edef\csname nodetime\theitemnumber \endcsname{#1}
}
% timeline scale and labels
\newcommand{\drawtimeline}[1]{% start date
    \draw[very thick,-latex] (0,0) -- ($(\lastnode.south west)-(\scaleitemseparation,0)+(0,-1)$);
    \ExtractCoordinate{n-\theitemnumber.south}
    \pgfmathsetmacro{\yposition}{\YCoord/28.452755}
    \foreach \x in {1,...,\theitemnumber}
    {   \pgfmathsetmacro{\timeposition}{\yposition/(\maxtime-\mintime)*\csname nodetime\x \endcsname}
        %\node[right] at (0,\timeposition) {\yposition};
        \draw (0,\timeposition) -- (0.5,\timeposition) -- ($(n-\x.west)-(0.5,0)$) -- (n-\x.west);
    }
    \foreach \x in {0,...,\timeintervals}
    {   \pgfmathsetmacro{\labelposition}{\yposition/(\maxtime-\mintime)*\x}
        \node[left] (label-\x) at (-0.2,\labelposition) {\textcolor{gray}{\bf \timeunit\ \the\numexpr\x + 1\relax} \ \DTMdate{#1 + \the\numexpr\x * 7\relax}\ -\ \DTMdate{#1 + \the\numexpr\x * 7 +6\relax}};
        \draw (label-\x.east) -- ++ (0.2,0.0);
    }   
}

% set date format as mm//dd
\renewcommand{\DTMdisplaydate}[4]{#2/#3}

\title{Surgical Phase Detection Using Deep Learning\\ Proposal \& Plan}
\author{Xiaorui Zhang, Wenkai Luo, Xucheng Ma}
\date{February 2022}

\begin{document}

\maketitle

\section{Stated topic and goal}
Surgical phase recognition plays a crucial role in the era of digitized surgery. Deep learning solutions have seen great success in endoscopic surgeries. Currently, no prior work has investigated its application in skull-base surgery (Cortical Mastoidectomy). This project will benchmark existing DL solutions and create an innovative DL segmentation algorithm in skull-based surgery.

\section{Team members, mentor}
\begin{itemize}
    \item \textbf{Students}:\\Xucheng Ma, Xiaorui Zhang, Wenkai Luo
    \item \textbf{Mentors}:\\Max Li, Danielle Trakimas, Dr.Francis Creighton, Prof. Mathias Unberath, Prof. Russ Taylor
\end{itemize}

\section{Relevance/importance}
Surgical phase recognition has numerous potential medical applications. Such as automatic indexing of surgical video databases, and optimization of real-time operating room scheduling. It’s also a foundation of intelligent context-aware system, which facilitates surgery monitoring, surgical protocol extraction, and decision support.
\section{Short technical summary of approach}
Kevin
\section{Deliverables}
Project deliverables are listed as follows:
\begin{itemize}
    \item \textbf{Minimum Deliverables}
    \begin{itemize}
        \item New dataset from cortical mastoidectomy videos (with Danielle's help)
        \item At least 3  methods​
        \item All methods trained and evaluated on new dataset​
    \end{itemize}
    \item \textbf{Expected Deliverables}
    \begin{itemize}
        \item Experiments and comparison with existing methods
        \item Ablation study
    \end{itemize}
    \item \textbf{Maximum Deliverables}
    \begin{itemize}
        \item Conference paper
    \end{itemize}
\end{itemize}

\newpage
\section{Timeline \& Milestones}
Figure \ref{fig:timeline} shows the project timeline. \textcolor{blue}{Milestones} are labeled in blue, and deadlines for resolving \textcolor{red}{dependencies} are labeled in red. Other entries in the timeline indicate either start point or end point of tasks.
\begin{figure}[!ht]
    \centering
    \begin{tikzpicture}
        \node[inner sep=0] (n-0) at (\scaleitemseparation,0) {};
        \timeentry{0.5}{Literature review}
        \timeentry{1.5}{Proposal and plan}
        \timeentry{2}{Start data annotations, environment setup}
        \timeentry{2.6}{\textcolor{blue}{\bf Proposal and Plan Presentation}}
        \timeentry{3.4}{\textcolor{red}{\bf Sample Dataset} }
        \timeentry{3.6}{Finish environment setup}
        \timeentry{4.5}{Start data-prepocession for \\ benchmark methods}
        \timeentry{5.5}{Start implementing model evaluation}
        \timeentry{6.5}{Ready to training benchmart methods}
        \timeentry{7.2}{\textcolor{red}{\bf Computational resources}}
        \timeentry{7.3}{\textcolor{red}{\bf Fullly annodated dataset}}
        \timeentry{7.4}{\textcolor{blue}{\bf Seminar Presentation}}
        \timeentry{7.6}{Benchmark methods training}
        \timeentry{8.5}{\textcolor{blue}{\bf Minimum Deliverables:\\ Benchmarking $\geq3$ methods}}
        \timeentry{9.3}{\textcolor{blue}{\bf Checkpoint Presentation}}
        \timeentry{9.7}{\textcolor{blue}{\bf Initial design of improved method}}
        \timeentry{10.5}{Train and evaluate improved method}
        \timeentry{11.5}{\textcolor{blue}{\bf Expected Deliverables:\\ Experiment results of improved method}}
        \timeentry{12.5}{Poster and final report}
        \timeentry{13.5}{Additional experiments}
        \timeentry{14}{\textcolor{blue}{\bf Final presentation}}
        \drawtimeline{2022-1-24}
    \end{tikzpicture}
    \caption{Project timeline, \textcolor{blue}{Milestones} labeled in \textcolor{blue}{blue}, \textcolor{red}{Deliverables deadline} labeled in \textcolor{red}{red}}
    \label{fig:timeline}
\end{figure}

% \begin{enumerate}
%     \item 01/24 - 01/30
%           \begin{itemize}
%               \item literature review
%           \end{itemize}
%     \item 01/31 - 02/06
%           \begin{itemize}
%               \item proposal
%               \item project plan
%           \end{itemize}
%     \item 02/07 - 02/13
%           \begin{itemize}
%               \item start working on:
%                     \begin{itemize}
%                         \item data annotations
%                         \item environment setup
%                         \item reading and running existing code
%                     \end{itemize}
%               \item proposal presentation 02/10
%           \end{itemize}
%     \item 02/14 - 02/20
%           \begin{itemize}
%               \item check points
%                     \begin{itemize}
%                         \item sample data set: at least one video, at least phase label for each frame
%                         \item reproduce EndoNet results on public laparoscopic dataset (Cholec80)
%                     \end{itemize}
%           \end{itemize}
%     \item 02/21 - 02/27
%           \begin{itemize}
%               \item check points
%                     \begin{itemize}
%                         \item more annotated data
%                     \end{itemize}
%               \item start working on
%                     \begin{itemize}
%                         \item data pre-processing for different existing methods
%                     \end{itemize}
%           \end{itemize}
%     \item 02/28 - 03/06
%           \begin{itemize}
%               \item start working on
%                     \begin{itemize}
%                         \item implementing evaluation functions
%                     \end{itemize}
%           \end{itemize}
%     \item 03/07 - 03/13
%           \begin{itemize}
%               \item checkpoints
%                     \begin{itemize}
%                         \item data pre-processing
%                         \item evaluation
%                     \end{itemize}
%           \end{itemize}
%     \item 03/14 - 03/20
%           \begin{itemize}
%               \item seminar presentation
%               \item benchmark methods training
%           \end{itemize}
%     \item 03/21 - 03/27(Spring Break)
%           \begin{itemize}
%               \item checkpoints
%                     \begin{itemize}
%                         \item minimum deliverable: benchmarks of at least 3 methods, evaluation and comparison
%                     \end{itemize}
%           \end{itemize}
%     \item 03/28 - 04/03
%           \begin{itemize}
%               \item checkpoint presentation
%               \item design improved network
%           \end{itemize}
%     \item 04/04 - 04/10
%           \begin{itemize}
%               \item start working on
%                     \begin{itemize}
%                         \item training improved network
%                         \item evaluating improved network
%                     \end{itemize}
%           \end{itemize}
%     \item 04/11 - 04/17
%           \begin{itemize}
%               \item checkpoints
%                     \begin{itemize}
%                         \item expected deliverable: experiment results
%                     \end{itemize}
%           \end{itemize}
%     \item 04/18 - 04/24
%           \begin{itemize}
%               \item Start working on
%                     \begin{itemize}
%                         \item poster
%                         \item final report
%                     \end{itemize}
%               \item additional experiment if needed
%           \end{itemize}
%     \item 04/25 - 05/01
%           \begin{itemize}
%               \item checkpoints
%                     \begin{itemize}
%                         \item poster
%                         \item final report
%                     \end{itemize}
%           \end{itemize}
% \end{enumerate}

\section{List of dependencies \& plan for resolving}
\begin{enumerate}
    \item Dataset \& Annotations
          \begin{enumerate}
              \item Dataset: Cortical mastoidectomy video dataset provided by JHMI $\rightarrow$ Need to talk to Dr. Danielle Trakimas to obtain videos.
              \item Annotation: Phase definition and annotation protocol $\rightarrow$ Need to talk to Dr. Danielle Trakimas to finalize the annotation protocol and start annotations.
          \end{enumerate}
    \item Computer, GPU, and server setup
          \begin{enumerate}
              \item Computer: Our personal computers with the required packages installed
                    \begin{enumerate}
                        \item Environment: Pytorch+torchvision+cuda
                        \item Internet connection to remotely access GPU server.
                    \end{enumerate}
              \item GPU resource: ARCADE Lab
                    \begin{enumerate}
                        \item ARCADE Lab access $\rightarrow$ Need to apply and obtain lab access approval from Prof. Mathias Unberath.
                        \item Remote server $\rightarrow$ Need to set up the remote GPU server service on the ARCADE Lab.
                    \end{enumerate}
          \end{enumerate}
    \item Existing Framework \& Public dataset
          \begin{enumerate}
              \item Existing Framework: Open source frameworks are available online from Paperwithcode and Github
                    \begin{enumerate}
                        \item EndoNet: Not available online but should be able to be reproduced.
                        \item MTRCNet-CL: \url{https://github.com/YuemingJin/MTRCNet-CL}
                        \item Trans-SVNet: \url{https://github.com/xjgaocs/Trans-SVNet}
                    \end{enumerate}
              \item Public dataset:
                    \begin{enumerate}
                        \item Cholec80 $\rightarrow$ Need to apply to CAMMA for dataset access. \\(\url{http://camma.u-strasbg.fr/datasets})
                        \item M2CAI 2016 Challenge Datasets $\rightarrow$ Need to apply to CAMMA for dataset access. \\(\url{http://camma.u-strasbg.fr/datasets})
                    \end{enumerate}
          \end{enumerate}
    \item Clinical advice
          \begin{enumerate}
              \item operation analysis
              \item experiment result analysis
          \end{enumerate}
          Need to obtain advice from Dr. Danielle Trakimas.
\end{enumerate}
\section{Management Plan}
Xucheng
\section{Reading list}
ALL

\end{document}
